\section{Introduction}
Despite the advent of more precise methods, the use of leaves to identify trees and other plants persists. The variance of shapes between species with a general uniformity of shape within one species, combined with the portability and size of the leaves, has ensured their popularity for this usage. 
Whereas leaf identification used to require manual comparison of the leaf (or a photograph of it) to images of previous leaves (often involving large decision trees), advances in computer vison have made fast, large scale comparison of images feasible \cite{Belh2008}. One example of the use of computer image classification for leafs is the Leaf Snap app \cite{Kuma2012}: a mobile app covering all 185 species of the northeast USA, it enables users to photograph leaves and immediately classify them, by transmitting the captures image to a server which houses the recognition system. After classification, the user is presented with a sorted list of identification results from which they can pick the one that most resembles their leaf. Total time to solution after uploading of the image is 5.4 seconds \cite{Kuma2012}.
We decided to develop a system with a similar goal, to identify the species of a leaf by using Machine Learning techniques to compare it with a processed database of labeled examples. The resulting system is capable of performing the whole process: feature extraction from an image database and using the extracted features to train a classifier. We used two available datasets, one of which was part of the ImageClef 2012 leaf classification challenge, earlier installments of which have been used before in comparable efforts to make an image classification system \cite{Goea2011}.

This is the introduction.
It is imported from the file \verb+introduction.tex+.

\subsection{Instructions}
\subsubsection{Making the PDF}
Run \verb+makePDF.sh+ (Linux, OS X) or \verb+makePDF.bat+ (Windows) in the terminal to make a pdf of the whole paper.
This script also takes care of changes to the bibliography, so you only need to run it once.

\subsubsection{Editing}
Each section of the paper has its own \verb+.tex+ file that is imported into the main \verb+ML_leaf_report.tex+ file, so the sections can be worked on simultaneously without messing up the main file (hopefully).

\subsubsection{Preamble}
Finally, the file \verb+preamble.sty+ contains all package imports in one place.

\subsection{Another subsection}
More text goes here.
Let's cite something~\cite{dijkstra68}.
Citing some more:~\cite{charniak85,steels98}.